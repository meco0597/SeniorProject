\documentclass[11pt]{ieeeconf}
\usepackage{graphicx}
\usepackage{float}
\usepackage{lettrine}
\usepackage{caption}
\usepackage{url}


\newcommand\blfootnote[1]{%
  \begingroup
  \renewcommand\thefootnote{}\footnote{#1}%
  \addtocounter{footnote}{-1}%
  \endgroup
}

\title{Bumper Car Sumo}
\author{Jaden Simon - simonjaden223@gmail.com \\ \and
	   Melvin Bosnjak - meco0597@gmail.com \\ \and
	   Daniel Humeniuk - d.humeniuk@utah.edu \\ \and
	   http://bumpercarsumo.weebly.com/}

%Herein is our final report

%Professor Stevens' requirements for the report bellow

%Make sure you take time for your technical documentation! The final technical report can be completed the week following your project demonstration. Don't forget to plan for that! The final technical document will be written in LaTeX and will require formatting as the two column IEEE conference or journal paper as was done in 3991.

%This aspect of the project is critical to ensure that you are communicating well with the instructor as to the status of your project. This portion of the grade will be based on the weekly logs and team meetings that you have with the instructor. It is critical that you understand the progress of your project and that you are able to communicate that to your team members and the instructor. This serves as an early warning system in such instances as when risks can not be reduced or worst case scenarios play out, when the required engineering effort was under estimated, or when parts don't arrive or a team member is not delivering or needs help to keep from becoming a roadblock to progress of the project. It takes discipline to manage a team, and this grade is part of that. You will be recorded 0.67% of your grade for each week's team log. If you don't turn one in, you will not get any points for that week. If the log is insufficient, you will get half credit for that week. A log is insufficient if it is not clear what progress has been made during the week, whether the project is on schedule, and what the current perceived risks are that could prevent delivering on the project specifications by demo day.

%Part of the reporting is meeting with the instructor. These meetings can be as often as every week, depending on the need of the team. Some of these meetings will be mini design reviews, others will be status and risk updates. Use the instructor and outside resources to help your project be a smashing success! Make sure you acknowledge the help of others in your technical report. 
	   
\begin{document}

\maketitle

\begin{abstract}
BumperCarSumo is a fully interactive multiplayer game. Each player controls a little robot on a designated play area. A robot is controlled by a central game hub that communicates to the robots via WiFi. The controllers are connected to the hub with the use of Bluetooth technology. The robots are powered internally with a rechargable lithium ion battery and moves about the play area with geared DC Motors. The goal of the game is to be the last remaining player on the play area by any means possible. Once a player has fallen off of the play area, the player has lost the match. The hub tracks each robot with a web camera and an OpenCV program to ensure that once it has fallen off the play area, the robot will turn off and remains off until the next game. Once all but one player remains, the round is over and is free to be played again.
\end{abstract}
%This is all taken from 2018 requirements of the report and will be updated as needed

\section{Introduction}
%Introduction and motivation (describe the problem, and who cares if you solve it)
The purpose of BumperCarSumo is to provide entertainment that mirrors popular Battle Royale game play with the use of physically controlled objects. The components currently required for game play is a white play area, a dark (black if possible) out of bounds zone, a web camera, a Raspberry Pi 3 B+, the BumperCarSumo program loaded onto the Raspberry Pi, two to four remote controlled robots, and the corresponding amount of controllers. Each component will be described throughout this report. 

\section{Background}

%Background (What other things are out there that you're drawing from? Are there similar things that you're referencing or extending?)

\section{Project Implementation}
%Project Implementation (The details of what you did, what you used, and why you made the choices you did. This is likely to be by far the largest section of your report, and can have multiple sub-sections) 
\subsection{The Game Hub}

We utilized a Raspberry Pi 3 B+ as our game hub. The device contained both Bluetooth and WiFi capabilities which was critical for communicating to the external game peripherals. The Pi was an excellent choice as it is small, portable, and capable enough to handle the software required for the game. 

The controllers were connected to the Game Hub via Bluetooth. The controllers are based on Wii Balance Boards. The Game Hub starts the controller logic in a separate thread and once connected, continuously polls the Bluetooth devices to ensure they stay connected. When a game is in progress, the data collected from Bluetooth is processed. Once processed, the Game Hub will determine in which direction the player is leaning and send the corresponding command to the player's robot.

\section{Discoveries}
%Discoveries and pitfalls (things you learned and things that other groups would find helpful)

\section{Evaluation}
%How well did it work - be honest here - if there were things that didn't work out the way you expected, write about them too. Also make sure to describe your testing and evaluation strategy. Don't just say it worked - describe how you know it worked, and even what "worked" means. This is likely to be the second-largest section in your report.

\subsection{Melvin}

\subsection{Jaden}

\subsection{Dan}

\section{Bill of Materials}
% Bill of materials (or at least a list of all the materials, equipment, etc. that you used. You don't need prices here, but you should include a list of things you used)

\section{Conclusion}
%- Conclusions (wrap things up with some final text. Include a pointer to your final web site.) 

\bibliographystyle{IEEEtran}
%\bibliography{IEEEabrv,bib/ref}

\end{document}